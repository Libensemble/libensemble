%%---------------------------------------------------------------------------%%
%% report.tex
%% Seth Johnson and Tom Evans
%% Modified by Todd Munson April 17, 2017
%% Modified by Todd Munson June  27, 2017
%% Copyright (C) 2008-2017 Oak Ridge National Laboratory, UT-Battelle, LLC.
%%---------------------------------------------------------------------------%%

% Add the 'draft' option to compile faster (without images e.g.)
\documentclass{ecpreport}

%%---------------------------------------------------------------------------%%
%% VARIABLES
%%---------------------------------------------------------------------------%%
\wbs{1.3.3.07}
\projecttitle{Preparing PETSc/TAO for Exascale}
\milestoneid{STMS07-8}
\title{Design and Prototype an API for libEnsemble}
\author{Jeffrey Larson (Argonne)
  \and  Todd Munson (Argonne)
  \and  Barry Smith (Argonne)
  \and  Stefan Wild (Argonne)
}
\date{\today}

%%---------------------------------------------------------------------------%%
%% FRONT MATTER
%%---------------------------------------------------------------------------%%

\begin{document}
\frontmatter

%%---------------------------------------------------------------------------%%
% EXECUTIVE SUMMARY
%%---------------------------------------------------------------------------%%

\begin{abstract}
The availability of systems with over 100 times the processing power of today’s 
machines compels the utilization of these systems not just for a single ``forward
solve'' simulation, but rather within a tight loop of optimization, sensitivity 
analysis (SA), and uncertain quantification (UQ). This requires the implementation 
of a new, scalable library for managing a dynamic hierarchical collection of 
running scalable simulations, where the simulations directly feed results 
into the optimization, SA, and UQ solvers.  The collection of running simulations 
can grow and shrink based on feedback from the solvers. Thus, this library must 
dynamically start simulations with different parameters, resume simulations to 
obtain more accurate results, prune running simulations that the solvers determine 
can no longer provide useful information, monitor the progress of the simulations 
and stop failed or hung simulations, and collect data from the individual 
simulations both while they are running and at the end.  This library, which 
we call {\bf libEnsemble} should not be confused with workflow-based scripting 
systems; rather it is a library that, through the tight coupling and feedback 
described above, directs the multiple concurrent ``function evaluations'' 
needed by optimization, SA, and UQ solvers. Currently no library is available 
with any robustness to failure of even a single simulation; meaning that current 
codes must be completely restarted if a single simulation in the ensemble 
crashes or hangs. This situation is clearly unacceptable at the exascale
and needs to be rectified quickly.  This report documents the initial
design and prototype for {\bf libEnsemble}.
\end{abstract}

%%---------------------------------------------------------------------------%%
% MAIN DOCUMENT
%%---------------------------------------------------------------------------%%

\mainmatter
\section{Introduction}

The availability of systems with over 100 times the processing power of today’s 
machines compels the utilization of these systems not just for a single ``forward 
solve'' simulation, but rather within a tight loop of optimization, sensitivity 
analysis (SA), and uncertain quantification (UQ). This requires the implementation 
of a new, scalable library for managing a dynamic hierarchical collection of 
running scalable simulations, where the simulations directly feed results 
into the optimization, SA, and UQ solvers.  The collection of running simulations 
can grow and shrink based on feedback from the solvers. Thus, this library must 
dynamically start simulations with different parameters, resume simulations to 
obtain more accurate results, prune running simulations that the solvers determine 
can no longer provide useful information, monitor the progress of the simulations 
and stop failed or hung simulations, and collect data from the individual 
simulations both while they are running and at the end.  This library, which 
we call {\bf libEnsemble} should not be confused with workflow-based scripting 
systems; rather it is a library that, through the tight coupling and feedback 
described above, directs the multiple concurrent ``function evaluations'' 
needed by optimization, SA, and UQ solvers. Currently no library is available 
with any robustness to failure of even a single simulation; meaning that current
codes must be completely restarted if a single simulation in the ensemble 
crashes or hangs. This situation is clearly unacceptable at the exascale
and needs to be rectified quickly.  This report documents the initial
design and prototype for {\bf libEnsemble}.

\section{Milestone Overview}

\subsection{Description}

Design and prototype an API for libEnsemble.

\subsection{Execution Plan}

\begin{itemize}
\item Identify initial use cases and requirements
\item Design and document libEnsemble API to satisfy initial requirements
\item Implement prototype version of libEnsemble API in Python
\item Test prototype API on small-scale example
\end{itemize}

\subsection{Completion Criteria}

\begin{itemize}
\item Requirements document for libEnsemble API
\item Prototype Python implementation in repository
\end{itemize}

\subsection{Milestone Dependencies}
\subsubsection{Milestone Predecessors}

None

\subsubsection{Milestone Successors}

STMS07-9, Provide a prototype libEnsemble with evaluations of problem usage

\section{Resource Requirements}

We estimate that this milestone used \$250K of the project funds to support
the effort of Jeff Larson, Todd Munson, and Stefan Wild.

\section{Technical Work Scope, Approach, and Results}

This milestone focused on producing a requirements document for {\bf libEnsemble},
developing a prototype implementation in python, and testing the prototype on
some simple test cases.  A snapshot of the requirements document is found
in Appendix~A; this document will evolve as {\bf libEnsemble} evolves.
The prototype implementation is found at 
\url{https://github.com/Libensemble/libensemble}.

\section{Conclusions and Future Work}

Our future work is to harden the code and release a prototype of the
basic libEnsemble to users, and evaluate the prototype using the 
POUNDerS optimization algorithm in TAO on parameter estimation 
test problems.

\section*{Acknowledgments}

This research was supported by the Exascale Computing Project (17-SC-20-SC), a collaborative 
effort of two U.S. Department of Energy organizations (Office of Science and the National 
Nuclear Security Administration) responsible for the planning and preparation of a 
capable exascale ecosystem, including software, applications, hardware, advanced 
system engineering, and early testbed platforms, in support of the nation’s 
exascale computing imperative.

\bibliographystyle{plain}
\bibliography{references}

%%---------------------------------------------------------------------------%%
%% APPENDIX
%%---------------------------------------------------------------------------%%

\newpage
\appendix
\section{Appendix A}
\includepdf[pages=-]{docs/planning_doc.pdf}

\end{document}

